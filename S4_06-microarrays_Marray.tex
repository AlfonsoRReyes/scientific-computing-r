\documentclass[]{article}
\usepackage{lmodern}
\usepackage{amssymb,amsmath}
\usepackage{ifxetex,ifluatex}
\usepackage{fixltx2e} % provides \textsubscript
\ifnum 0\ifxetex 1\fi\ifluatex 1\fi=0 % if pdftex
  \usepackage[T1]{fontenc}
  \usepackage[utf8]{inputenc}
\else % if luatex or xelatex
  \ifxetex
    \usepackage{mathspec}
  \else
    \usepackage{fontspec}
  \fi
  \defaultfontfeatures{Ligatures=TeX,Scale=MatchLowercase}
\fi
% use upquote if available, for straight quotes in verbatim environments
\IfFileExists{upquote.sty}{\usepackage{upquote}}{}
% use microtype if available
\IfFileExists{microtype.sty}{%
\usepackage{microtype}
\UseMicrotypeSet[protrusion]{basicmath} % disable protrusion for tt fonts
}{}
\usepackage[margin=1in]{geometry}
\usepackage{hyperref}
\hypersetup{unicode=true,
            pdftitle={Microarrays, Marray},
            pdfborder={0 0 0},
            breaklinks=true}
\urlstyle{same}  % don't use monospace font for urls
\usepackage{color}
\usepackage{fancyvrb}
\newcommand{\VerbBar}{|}
\newcommand{\VERB}{\Verb[commandchars=\\\{\}]}
\DefineVerbatimEnvironment{Highlighting}{Verbatim}{commandchars=\\\{\}}
% Add ',fontsize=\small' for more characters per line
\usepackage{framed}
\definecolor{shadecolor}{RGB}{248,248,248}
\newenvironment{Shaded}{\begin{snugshade}}{\end{snugshade}}
\newcommand{\AlertTok}[1]{\textcolor[rgb]{0.94,0.16,0.16}{#1}}
\newcommand{\AnnotationTok}[1]{\textcolor[rgb]{0.56,0.35,0.01}{\textbf{\textit{#1}}}}
\newcommand{\AttributeTok}[1]{\textcolor[rgb]{0.77,0.63,0.00}{#1}}
\newcommand{\BaseNTok}[1]{\textcolor[rgb]{0.00,0.00,0.81}{#1}}
\newcommand{\BuiltInTok}[1]{#1}
\newcommand{\CharTok}[1]{\textcolor[rgb]{0.31,0.60,0.02}{#1}}
\newcommand{\CommentTok}[1]{\textcolor[rgb]{0.56,0.35,0.01}{\textit{#1}}}
\newcommand{\CommentVarTok}[1]{\textcolor[rgb]{0.56,0.35,0.01}{\textbf{\textit{#1}}}}
\newcommand{\ConstantTok}[1]{\textcolor[rgb]{0.00,0.00,0.00}{#1}}
\newcommand{\ControlFlowTok}[1]{\textcolor[rgb]{0.13,0.29,0.53}{\textbf{#1}}}
\newcommand{\DataTypeTok}[1]{\textcolor[rgb]{0.13,0.29,0.53}{#1}}
\newcommand{\DecValTok}[1]{\textcolor[rgb]{0.00,0.00,0.81}{#1}}
\newcommand{\DocumentationTok}[1]{\textcolor[rgb]{0.56,0.35,0.01}{\textbf{\textit{#1}}}}
\newcommand{\ErrorTok}[1]{\textcolor[rgb]{0.64,0.00,0.00}{\textbf{#1}}}
\newcommand{\ExtensionTok}[1]{#1}
\newcommand{\FloatTok}[1]{\textcolor[rgb]{0.00,0.00,0.81}{#1}}
\newcommand{\FunctionTok}[1]{\textcolor[rgb]{0.00,0.00,0.00}{#1}}
\newcommand{\ImportTok}[1]{#1}
\newcommand{\InformationTok}[1]{\textcolor[rgb]{0.56,0.35,0.01}{\textbf{\textit{#1}}}}
\newcommand{\KeywordTok}[1]{\textcolor[rgb]{0.13,0.29,0.53}{\textbf{#1}}}
\newcommand{\NormalTok}[1]{#1}
\newcommand{\OperatorTok}[1]{\textcolor[rgb]{0.81,0.36,0.00}{\textbf{#1}}}
\newcommand{\OtherTok}[1]{\textcolor[rgb]{0.56,0.35,0.01}{#1}}
\newcommand{\PreprocessorTok}[1]{\textcolor[rgb]{0.56,0.35,0.01}{\textit{#1}}}
\newcommand{\RegionMarkerTok}[1]{#1}
\newcommand{\SpecialCharTok}[1]{\textcolor[rgb]{0.00,0.00,0.00}{#1}}
\newcommand{\SpecialStringTok}[1]{\textcolor[rgb]{0.31,0.60,0.02}{#1}}
\newcommand{\StringTok}[1]{\textcolor[rgb]{0.31,0.60,0.02}{#1}}
\newcommand{\VariableTok}[1]{\textcolor[rgb]{0.00,0.00,0.00}{#1}}
\newcommand{\VerbatimStringTok}[1]{\textcolor[rgb]{0.31,0.60,0.02}{#1}}
\newcommand{\WarningTok}[1]{\textcolor[rgb]{0.56,0.35,0.01}{\textbf{\textit{#1}}}}
\usepackage{graphicx,grffile}
\makeatletter
\def\maxwidth{\ifdim\Gin@nat@width>\linewidth\linewidth\else\Gin@nat@width\fi}
\def\maxheight{\ifdim\Gin@nat@height>\textheight\textheight\else\Gin@nat@height\fi}
\makeatother
% Scale images if necessary, so that they will not overflow the page
% margins by default, and it is still possible to overwrite the defaults
% using explicit options in \includegraphics[width, height, ...]{}
\setkeys{Gin}{width=\maxwidth,height=\maxheight,keepaspectratio}
\IfFileExists{parskip.sty}{%
\usepackage{parskip}
}{% else
\setlength{\parindent}{0pt}
\setlength{\parskip}{6pt plus 2pt minus 1pt}
}
\setlength{\emergencystretch}{3em}  % prevent overfull lines
\providecommand{\tightlist}{%
  \setlength{\itemsep}{0pt}\setlength{\parskip}{0pt}}
\setcounter{secnumdepth}{0}
% Redefines (sub)paragraphs to behave more like sections
\ifx\paragraph\undefined\else
\let\oldparagraph\paragraph
\renewcommand{\paragraph}[1]{\oldparagraph{#1}\mbox{}}
\fi
\ifx\subparagraph\undefined\else
\let\oldsubparagraph\subparagraph
\renewcommand{\subparagraph}[1]{\oldsubparagraph{#1}\mbox{}}
\fi

%%% Use protect on footnotes to avoid problems with footnotes in titles
\let\rmarkdownfootnote\footnote%
\def\footnote{\protect\rmarkdownfootnote}

%%% Change title format to be more compact
\usepackage{titling}

% Create subtitle command for use in maketitle
\newcommand{\subtitle}[1]{
  \posttitle{
    \begin{center}\large#1\end{center}
    }
}

\setlength{\droptitle}{-2em}

  \title{Microarrays, Marray}
    \pretitle{\vspace{\droptitle}\centering\huge}
  \posttitle{\par}
    \author{}
    \preauthor{}\postauthor{}
    \date{}
    \predate{}\postdate{}
  

\begin{document}
\maketitle

Source: \url{https://github.com/lgatto/S4-tutorial}

We choose to represent the microarray results as a matrix of size n×m,
where n is the number of probes on the microarray and m is the number of
samples. The matrix that stores the intensities (these could also be
fold-changes) is named marray.

The sample annotation (meta-data) is described using a data.frame with
exactly m rows and any number of columns. It is named pmeta.

The feature (probe) annotation (meta-data) is described using a
data.frame with exactly n rows and any number of columns. Let's call it
fmeta.

\begin{Shaded}
\begin{Highlighting}[]
\NormalTok{n <-}\StringTok{ }\DecValTok{10}
\NormalTok{m <-}\StringTok{ }\DecValTok{6}

\NormalTok{marray <-}\StringTok{ }\KeywordTok{matrix}\NormalTok{(}\KeywordTok{rnorm}\NormalTok{(n }\OperatorTok{*}\StringTok{ }\NormalTok{m, }\DecValTok{10}\NormalTok{, }\DecValTok{5}\NormalTok{), }\DataTypeTok{ncol =}\NormalTok{ m)}

\CommentTok{# add the row names and column names to the matrix}
\KeywordTok{rownames}\NormalTok{(marray) <-}\StringTok{ }\KeywordTok{paste0}\NormalTok{(}\StringTok{"probe"}\NormalTok{, }\DecValTok{1}\OperatorTok{:}\NormalTok{n)}
\KeywordTok{colnames}\NormalTok{(marray) <-}\StringTok{ }\NormalTok{LETTERS[}\DecValTok{1}\OperatorTok{:}\NormalTok{m]}

\NormalTok{marray}
\end{Highlighting}
\end{Shaded}

\begin{verbatim}
##                  A         B         C         D         E         F
## probe1   6.1924076  7.217565  5.790635 11.995220 10.101586  9.303057
## probe2  -0.1535608  7.009127 12.025718 14.221536  5.792542 12.385899
## probe3  12.0588658  6.547206 11.572435 10.161414  5.708033  6.988485
## probe4  17.2480906 15.162844 -1.348905 12.002849 14.989759 17.283304
## probe5   4.8047558 11.811113  6.274948 17.640637  8.925632 14.956560
## probe6   6.3817510 18.338020 18.249682 10.345714  6.816407 18.414272
## probe7  12.3156504  8.491294  8.978839 12.917494  6.389150  5.198944
## probe8  11.3835331  9.226863  9.309726 10.625132 11.226183  6.791711
## probe9   5.1553924 12.231468 15.940840 13.348883  8.885494  1.235753
## probe10 14.3927052 13.934608 13.744562  6.477938  6.546492 17.502218
\end{verbatim}

\hypertarget{create-dataframes-pmeta-and-fmeta}{%
\subsubsection{create dataframes pmeta and
fmeta}\label{create-dataframes-pmeta-and-fmeta}}

\begin{Shaded}
\begin{Highlighting}[]
\CommentTok{# build the annotations dataframe for the samples}
\NormalTok{pmeta <-}\StringTok{ }\KeywordTok{data.frame}\NormalTok{(}\DataTypeTok{sampleId =} \DecValTok{1}\OperatorTok{:}\NormalTok{m, }\DataTypeTok{condition =} \KeywordTok{rep}\NormalTok{(}\KeywordTok{c}\NormalTok{(}\StringTok{"WT"}\NormalTok{, }\StringTok{"MUT"}\NormalTok{), }\DataTypeTok{each =} \DecValTok{3}\NormalTok{))}
\KeywordTok{rownames}\NormalTok{(pmeta) <-}\StringTok{ }\KeywordTok{colnames}\NormalTok{(marray)}

\NormalTok{pmeta}
\end{Highlighting}
\end{Shaded}

\begin{verbatim}
##   sampleId condition
## A        1        WT
## B        2        WT
## C        3        WT
## D        4       MUT
## E        5       MUT
## F        6       MUT
\end{verbatim}

\begin{Shaded}
\begin{Highlighting}[]
\CommentTok{# build the annotations dataframe for the probes}
\NormalTok{fmeta <-}\StringTok{ }\KeywordTok{data.frame}\NormalTok{(}\DataTypeTok{geneId =} \DecValTok{1}\OperatorTok{:}\NormalTok{n, }\DataTypeTok{pathway =} \KeywordTok{sample}\NormalTok{(LETTERS, n, }\DataTypeTok{replace =} \OtherTok{TRUE}\NormalTok{))}
\KeywordTok{rownames}\NormalTok{(fmeta) <-}\StringTok{ }\KeywordTok{rownames}\NormalTok{(marray)}
\NormalTok{fmeta}
\end{Highlighting}
\end{Shaded}

\begin{verbatim}
##         geneId pathway
## probe1       1       L
## probe2       2       Z
## probe3       3       A
## probe4       4       D
## probe5       5       Z
## probe6       6       M
## probe7       7       O
## probe8       8       P
## probe9       9       A
## probe10     10       P
\end{verbatim}

Finally, to link these related pieces of information together, marray,
pmeta and fmeta will all be combined into a list that will represent our
microarray experiment.

\begin{Shaded}
\begin{Highlighting}[]
\NormalTok{maexp <-}\StringTok{ }\KeywordTok{list}\NormalTok{(}\DataTypeTok{marray =}\NormalTok{ marray, }
              \DataTypeTok{fmeta =}\NormalTok{ fmeta, }
              \DataTypeTok{pmeta =}\NormalTok{ pmeta)}


\KeywordTok{rm}\NormalTok{(marray, fmeta, pmeta) ## clean up}
\KeywordTok{str}\NormalTok{(maexp)}
\end{Highlighting}
\end{Shaded}

\begin{verbatim}
## List of 3
##  $ marray: num [1:10, 1:6] 6.192 -0.154 12.059 17.248 4.805 ...
##   ..- attr(*, "dimnames")=List of 2
##   .. ..$ : chr [1:10] "probe1" "probe2" "probe3" "probe4" ...
##   .. ..$ : chr [1:6] "A" "B" "C" "D" ...
##  $ fmeta :'data.frame':  10 obs. of  2 variables:
##   ..$ geneId : int [1:10] 1 2 3 4 5 6 7 8 9 10
##   ..$ pathway: Factor w/ 7 levels "A","D","L","M",..: 3 7 1 2 7 4 5 6 1 6
##  $ pmeta :'data.frame':  6 obs. of  2 variables:
##   ..$ sampleId : int [1:6] 1 2 3 4 5 6
##   ..$ condition: Factor w/ 2 levels "MUT","WT": 2 2 2 1 1 1
\end{verbatim}

We can access and manipulate the respective elements of our microarray
experi- ment with the \texttt{\$} operator.

\begin{Shaded}
\begin{Highlighting}[]
\NormalTok{maexp}\OperatorTok{$}\NormalTok{pmeta}
\end{Highlighting}
\end{Shaded}

\begin{verbatim}
##   sampleId condition
## A        1        WT
## B        2        WT
## C        3        WT
## D        4       MUT
## E        5       MUT
## F        6       MUT
\end{verbatim}

\begin{Shaded}
\begin{Highlighting}[]
\CommentTok{# get the summary for column A in the array}
\KeywordTok{summary}\NormalTok{(maexp}\OperatorTok{$}\NormalTok{marray[, }\StringTok{"A"}\NormalTok{])}
\end{Highlighting}
\end{Shaded}

\begin{verbatim}
##    Min. 1st Qu.  Median    Mean 3rd Qu.    Max. 
## -0.1536  5.4146  8.8826  8.9780 12.2515 17.2481
\end{verbatim}

\begin{Shaded}
\begin{Highlighting}[]
\CommentTok{# for the condition "WT" in the sample annotations, }
\CommentTok{# get the array values for probe8}
\NormalTok{wt <-}\StringTok{ }\NormalTok{maexp}\OperatorTok{$}\NormalTok{pmeta[, }\StringTok{"condition"}\NormalTok{] }\OperatorTok{==}\StringTok{ "WT"}
\NormalTok{maexp}\OperatorTok{$}\NormalTok{marray[}\StringTok{"probe8"}\NormalTok{, wt]}
\end{Highlighting}
\end{Shaded}

\begin{verbatim}
##         A         B         C 
## 11.383533  9.226863  9.309726
\end{verbatim}

\begin{Shaded}
\begin{Highlighting}[]
\CommentTok{# get values from probe3 for conditions different that "WT"}
\NormalTok{ maexp[[}\StringTok{"marray"}\NormalTok{]][}\StringTok{"probe3"}\NormalTok{, }\OperatorTok{!}\NormalTok{wt] ## different syntax}
\end{Highlighting}
\end{Shaded}

\begin{verbatim}
##         D         E         F 
## 10.161414  5.708033  6.988485
\end{verbatim}

The above solution does not provide a clean syntax. As a user, we have
to know the names or positions of the respective elements of the
microarray list elements to directly access the parts of interest.

\begin{Shaded}
\begin{Highlighting}[]
 \KeywordTok{boxplot}\NormalTok{(maexp}\OperatorTok{$}\NormalTok{marray)}
\end{Highlighting}
\end{Shaded}

\includegraphics{S4_06-microarrays_Marray_files/figure-latex/unnamed-chunk-9-1.pdf}

Figure 1: Boxplot representing the intensity distributions of the 10
probes for the 6 samples.

\hypertarget{challenge}{%
\subsection{Challenge}\label{challenge}}

But what if we want to subset the experiment. How would we extract the 5
first probes for the 3 first samples?

We have to manually subset the individual elements of our list, making
sure that the number of rows of the marray and fmeta elements remain
identical as well as the number of columns of marray and the number of
columns of pmeta.

\begin{Shaded}
\begin{Highlighting}[]
\NormalTok{x <-}\StringTok{ }\DecValTok{1}\OperatorTok{:}\DecValTok{5}
\NormalTok{y <-}\StringTok{ }\DecValTok{1}\OperatorTok{:}\DecValTok{3}

\NormalTok{marray2 <-}\StringTok{ }\NormalTok{maexp}\OperatorTok{$}\NormalTok{marray[x, y]}

\NormalTok{fmeta2 <-}\StringTok{ }\NormalTok{maexp}\OperatorTok{$}\NormalTok{fmeta[x, ]}
\NormalTok{pmeta2 <-}\StringTok{ }\NormalTok{maexp}\OperatorTok{$}\NormalTok{pmeta[y, ]}

\NormalTok{maexp2 <-}\StringTok{ }\KeywordTok{list}\NormalTok{(}\DataTypeTok{marray =}\NormalTok{ marray2, }\DataTypeTok{fmeta =}\NormalTok{ fmeta2, }\DataTypeTok{pmeta =}\NormalTok{ pmeta2)}

\KeywordTok{rm}\NormalTok{(marray2, fmeta2, pmeta2) ## clean up}
\KeywordTok{str}\NormalTok{(maexp2)}
\end{Highlighting}
\end{Shaded}

\begin{verbatim}
## List of 3
##  $ marray: num [1:5, 1:3] 6.192 -0.154 12.059 17.248 4.805 ...
##   ..- attr(*, "dimnames")=List of 2
##   .. ..$ : chr [1:5] "probe1" "probe2" "probe3" "probe4" ...
##   .. ..$ : chr [1:3] "A" "B" "C"
##  $ fmeta :'data.frame':  5 obs. of  2 variables:
##   ..$ geneId : int [1:5] 1 2 3 4 5
##   ..$ pathway: Factor w/ 7 levels "A","D","L","M",..: 3 7 1 2 7
##  $ pmeta :'data.frame':  3 obs. of  2 variables:
##   ..$ sampleId : int [1:3] 1 2 3
##   ..$ condition: Factor w/ 2 levels "MUT","WT": 2 2 2
\end{verbatim}

A simple operation like sub-setting the microarray experiment is very
cumbersome and prone to errors. If we were to use this implementation
for further work, we would of course want to write a custom function to
perform the above.

\hypertarget{the-marray-class}{%
\subsection{The MArray class}\label{the-marray-class}}

We can define a class with the setClass function. Our class is defined by
a name, MArray, and a content structure. The different elements/fields of
an S4 class are called slots4. When defining the slots, we provide their
respective names and classes as a (named) vector or list. It will only
be possible to create objects with exactly these types of slots.

\begin{Shaded}
\begin{Highlighting}[]
\NormalTok{MArray <-}\StringTok{ }\KeywordTok{setClass}\NormalTok{(}\StringTok{"MArray"}\NormalTok{,}
                   \DataTypeTok{slots =} \KeywordTok{c}\NormalTok{(}\DataTypeTok{marray =} \StringTok{"matrix"}\NormalTok{,}
                             \DataTypeTok{fmeta =} \StringTok{"data.frame"}\NormalTok{,}
                             \DataTypeTok{pmeta =} \StringTok{"data.frame"}\NormalTok{)}
\NormalTok{                       )}
\end{Highlighting}
\end{Shaded}

\begin{quote}
the representation function is deprecated from version 3.0.0 and should
be avoided.
\end{quote}

The setClass function returns a special function called a constructor,
that can be used to create an instance of the class.

\begin{Shaded}
\begin{Highlighting}[]
\KeywordTok{MArray}\NormalTok{() ## an empty object}
\end{Highlighting}
\end{Shaded}

\begin{verbatim}
## An object of class "MArray"
## Slot "marray":
## <0 x 0 matrix>
## 
## Slot "fmeta":
## data frame with 0 columns and 0 rows
## 
## Slot "pmeta":
## data frame with 0 columns and 0 rows
\end{verbatim}

\begin{Shaded}
\begin{Highlighting}[]
\CommentTok{# we use the previous list}
\NormalTok{ma <-}\StringTok{ }\KeywordTok{MArray}\NormalTok{(}\DataTypeTok{marray =}\NormalTok{ maexp[[}\DecValTok{1}\NormalTok{]],}
             \DataTypeTok{pmeta =}\NormalTok{ maexp[[}\StringTok{"pmeta"}\NormalTok{]],}
             \DataTypeTok{fmeta =}\NormalTok{ maexp[[}\StringTok{"fmeta"}\NormalTok{]]}
\NormalTok{             )}
\NormalTok{ma}
\end{Highlighting}
\end{Shaded}

\begin{verbatim}
## An object of class "MArray"
## Slot "marray":
##                  A         B         C         D         E         F
## probe1   6.1924076  7.217565  5.790635 11.995220 10.101586  9.303057
## probe2  -0.1535608  7.009127 12.025718 14.221536  5.792542 12.385899
## probe3  12.0588658  6.547206 11.572435 10.161414  5.708033  6.988485
## probe4  17.2480906 15.162844 -1.348905 12.002849 14.989759 17.283304
## probe5   4.8047558 11.811113  6.274948 17.640637  8.925632 14.956560
## probe6   6.3817510 18.338020 18.249682 10.345714  6.816407 18.414272
## probe7  12.3156504  8.491294  8.978839 12.917494  6.389150  5.198944
## probe8  11.3835331  9.226863  9.309726 10.625132 11.226183  6.791711
## probe9   5.1553924 12.231468 15.940840 13.348883  8.885494  1.235753
## probe10 14.3927052 13.934608 13.744562  6.477938  6.546492 17.502218
## 
## Slot "fmeta":
##         geneId pathway
## probe1       1       L
## probe2       2       Z
## probe3       3       A
## probe4       4       D
## probe5       5       Z
## probe6       6       M
## probe7       7       O
## probe8       8       P
## probe9       9       A
## probe10     10       P
## 
## Slot "pmeta":
##   sampleId condition
## A        1        WT
## B        2        WT
## C        3        WT
## D        4       MUT
## E        5       MUT
## F        6       MUT
\end{verbatim}

\begin{Shaded}
\begin{Highlighting}[]
\KeywordTok{class}\NormalTok{(ma)}
\end{Highlighting}
\end{Shaded}

\begin{verbatim}
## [1] "MArray"
## attr(,"package")
## [1] ".GlobalEnv"
\end{verbatim}

To access individual slots, we need to use the @. This is equivalent to
using the \texttt{\$} for a list.

\begin{Shaded}
\begin{Highlighting}[]
\NormalTok{ma}\OperatorTok{@}\NormalTok{pmeta}
\end{Highlighting}
\end{Shaded}

\begin{verbatim}
##   sampleId condition
## A        1        WT
## B        2        WT
## C        3        WT
## D        4       MUT
## E        5       MUT
## F        6       MUT
\end{verbatim}

But this is something we do not want a user to do. To access a slot like
this, one needs to know its name, i.e.~the underlying plumbing of the
class. This breaks the notion of encapsulation. Instead, the developer
will provide the user with specific accessor methods (see section 5.2) to
extract (or update using a replace method, section 5.5) specific slots.

\hypertarget{marray-methods}{%
\subsection{MArray methods}\label{marray-methods}}

Before proceeding, we need to explain the concept of generic function. A
generic function, or generic for short, is a function that dispatches
methods to their appro- priate class-specific implementation. A method do
will implement behaviour for a specific class A, while another
implementation of do, will define another behaviour for class B. The
generic do is the link between the class and its dedicated implemen-
tation. If we have do(a) (where a is of class A), than the generic will
make sure that the A-specific code of do will be executed.

Before we define a method with setMethod, we will always want to first
check if such a method does not exists (in which case there is already a
generic function), as illustrated with the show method in section 5.1.
If it is the case, we write our new methods. If not, we first create the
generic and then proceed with the method.

\hypertarget{the-show-method}{%
\subsection{The show method}\label{the-show-method}}

The show method (it is a method, as it exhibits custom behaviour
depending on the class of its argument) is a very helpful one. It allows
to define custom summary view of an object when we type its name in the
console, instead of having all its (possibly very long content)
displayed.

\begin{Shaded}
\begin{Highlighting}[]
\KeywordTok{setMethod}\NormalTok{(}\StringTok{"show"}\NormalTok{, }
          \DataTypeTok{signature =} \StringTok{"MArray"}\NormalTok{, }
          \DataTypeTok{definition =} \ControlFlowTok{function}\NormalTok{(object) \{}
            \KeywordTok{cat}\NormalTok{(}\StringTok{"An object of class "}\NormalTok{, }\KeywordTok{class}\NormalTok{(object), }\StringTok{"}\CharTok{\textbackslash{}n}\StringTok{"}\NormalTok{, }\DataTypeTok{sep =} \StringTok{""}\NormalTok{)}
            \KeywordTok{cat}\NormalTok{(}\StringTok{" "}\NormalTok{, }\KeywordTok{nrow}\NormalTok{(object}\OperatorTok{@}\NormalTok{marray), }\StringTok{" features by "}\NormalTok{, }\KeywordTok{ncol}\NormalTok{(object}\OperatorTok{@}\NormalTok{marray), }
                \StringTok{" samples.}\CharTok{\textbackslash{}n}\StringTok{"}\NormalTok{, }\DataTypeTok{sep =} \StringTok{""}\NormalTok{)}
            \CommentTok{#invisible(NULL) }
\NormalTok{            \})}
\end{Highlighting}
\end{Shaded}

\begin{Shaded}
\begin{Highlighting}[]
\NormalTok{ma}
\end{Highlighting}
\end{Shaded}

\begin{verbatim}
## An object of class MArray
##  10 features by 6 samples.
\end{verbatim}

\hypertarget{accessors}{%
\subsection{5.2 Accessors}\label{accessors}}

As mentioned above, we want to provide customised and controlled access
to the class slots. This does not prevent us, as developers, to use the
@ accessor, but does not force others to know the implementation
details. Let's create an accessor for the marray slot and call the
accessor marray. There is no harm in naming the slot and its accessor
with the same name but there is no constrain in doing so. There is no
such method or generic; just typing marray with tell you that no such
object is found. Below, we create a new generic function with
setGeneric. We define the name of our new generic as well as the name of
the argument(s) that will have to be re-used when defining class-specific
method.

\begin{Shaded}
\begin{Highlighting}[]
\KeywordTok{setGeneric}\NormalTok{(}\StringTok{"marray"}\NormalTok{, }\ControlFlowTok{function}\NormalTok{(object) }\KeywordTok{standardGeneric}\NormalTok{(}\StringTok{"marray"}\NormalTok{))}
\end{Highlighting}
\end{Shaded}

\begin{verbatim}
## [1] "marray"
\end{verbatim}

In general, it is considered good practice to add a \ldots{} in the
signature of a generic function. It provides the flexibility for other
methods to use more arguments.

\begin{Shaded}
\begin{Highlighting}[]
\KeywordTok{setGeneric}\NormalTok{(}\StringTok{"marray"}\NormalTok{, }\ControlFlowTok{function}\NormalTok{(object, ...) }\KeywordTok{standardGeneric}\NormalTok{(}\StringTok{"marray"}\NormalTok{))}
\end{Highlighting}
\end{Shaded}

\begin{verbatim}
## [1] "marray"
\end{verbatim}

We now proceed in the same way as above, using setMethod. The definition
of our method (i.e.~the actual code that will be executed) is very short
and of course uses @ to access (and return) the slot content.

\begin{Shaded}
\begin{Highlighting}[]
\KeywordTok{setMethod}\NormalTok{(}\StringTok{"marray"}\NormalTok{, }\StringTok{"MArray"}\NormalTok{, }\ControlFlowTok{function}\NormalTok{(object) object}\OperatorTok{@}\NormalTok{marray)}
\end{Highlighting}
\end{Shaded}

\begin{Shaded}
\begin{Highlighting}[]
\KeywordTok{marray}\NormalTok{(ma)}
\end{Highlighting}
\end{Shaded}

\begin{verbatim}
##                  A         B         C         D         E         F
## probe1   6.1924076  7.217565  5.790635 11.995220 10.101586  9.303057
## probe2  -0.1535608  7.009127 12.025718 14.221536  5.792542 12.385899
## probe3  12.0588658  6.547206 11.572435 10.161414  5.708033  6.988485
## probe4  17.2480906 15.162844 -1.348905 12.002849 14.989759 17.283304
## probe5   4.8047558 11.811113  6.274948 17.640637  8.925632 14.956560
## probe6   6.3817510 18.338020 18.249682 10.345714  6.816407 18.414272
## probe7  12.3156504  8.491294  8.978839 12.917494  6.389150  5.198944
## probe8  11.3835331  9.226863  9.309726 10.625132 11.226183  6.791711
## probe9   5.1553924 12.231468 15.940840 13.348883  8.885494  1.235753
## probe10 14.3927052 13.934608 13.744562  6.477938  6.546492 17.502218
\end{verbatim}

If we change the underlying implementation by changing the name of the
slot or using an environment instead of a matrix, the
\href{mailto:ma@marrayis}{\nolinkurl{ma@marrayis}} going to break.
However, when providing accessors, we can echo the changes in the
accessor implementation without affecting the users' behaviour or
existing scripts.

\hypertarget{accessors-for-pmeta-and-fmeta}{%
\subsection{Accessors for pmeta and
fmeta}\label{accessors-for-pmeta-and-fmeta}}

\begin{Shaded}
\begin{Highlighting}[]
\KeywordTok{setGeneric}\NormalTok{(}\StringTok{"pmeta"}\NormalTok{, }\ControlFlowTok{function}\NormalTok{(object, ...) }\KeywordTok{standardGeneric}\NormalTok{(}\StringTok{"pmeta"}\NormalTok{))}
\end{Highlighting}
\end{Shaded}

\begin{verbatim}
## [1] "pmeta"
\end{verbatim}

\begin{Shaded}
\begin{Highlighting}[]
\KeywordTok{setMethod}\NormalTok{(}\StringTok{"pmeta"}\NormalTok{, }\StringTok{"MArray"}\NormalTok{, }\ControlFlowTok{function}\NormalTok{(object) object}\OperatorTok{@}\NormalTok{pmeta)}

\KeywordTok{setGeneric}\NormalTok{(}\StringTok{"fmeta"}\NormalTok{, }\ControlFlowTok{function}\NormalTok{(object, ...) }\KeywordTok{standardGeneric}\NormalTok{(}\StringTok{"fmeta"}\NormalTok{))}
\end{Highlighting}
\end{Shaded}

\begin{verbatim}
## [1] "fmeta"
\end{verbatim}

\begin{Shaded}
\begin{Highlighting}[]
\KeywordTok{setMethod}\NormalTok{(}\StringTok{"fmeta"}\NormalTok{, }\StringTok{"MArray"}\NormalTok{, }\ControlFlowTok{function}\NormalTok{(object) object}\OperatorTok{@}\NormalTok{fmeta)}
\end{Highlighting}
\end{Shaded}

\begin{Shaded}
\begin{Highlighting}[]
\KeywordTok{pmeta}\NormalTok{(ma)}
\end{Highlighting}
\end{Shaded}

\begin{verbatim}
##   sampleId condition
## A        1        WT
## B        2        WT
## C        3        WT
## D        4       MUT
## E        5       MUT
## F        6       MUT
\end{verbatim}

\begin{Shaded}
\begin{Highlighting}[]
\KeywordTok{fmeta}\NormalTok{(ma)}
\end{Highlighting}
\end{Shaded}

\begin{verbatim}
##         geneId pathway
## probe1       1       L
## probe2       2       Z
## probe3       3       A
## probe4       4       D
## probe5       5       Z
## probe6       6       M
## probe7       7       O
## probe8       8       P
## probe9       9       A
## probe10     10       P
\end{verbatim}

\hypertarget{the-sub-setting-operation}{%
\subsection{The sub-setting operation}\label{the-sub-setting-operation}}

The documentation help(``{[}'') shows that, in addition to x, the object
to be subset, we also have to take the i and j indices into account and
the drop argument. When an argument is not relevant, we specify this by
declaring that it is ``missing''.

\begin{Shaded}
\begin{Highlighting}[]
\KeywordTok{setMethod}\NormalTok{(}\StringTok{"["}\NormalTok{, }\StringTok{"MArray"}\NormalTok{, }
          \ControlFlowTok{function}\NormalTok{(x, i, j, }\DataTypeTok{drop =} \StringTok{"missing"}\NormalTok{) \{ }
\NormalTok{              .marray <-}\StringTok{ }\NormalTok{x}\OperatorTok{@}\NormalTok{marray[i, j] }
\NormalTok{              .pmeta <-}\StringTok{ }\NormalTok{x}\OperatorTok{@}\NormalTok{pmeta[j, ] }
\NormalTok{              .fmeta <-}\StringTok{ }\NormalTok{x}\OperatorTok{@}\NormalTok{fmeta[i, ]}
                \KeywordTok{MArray}\NormalTok{(}\DataTypeTok{marray =}\NormalTok{ .marray,}
                        \DataTypeTok{fmeta =}\NormalTok{ .fmeta,}
                        \DataTypeTok{pmeta =}\NormalTok{ .pmeta)}
\NormalTok{\})}
\end{Highlighting}
\end{Shaded}

\begin{Shaded}
\begin{Highlighting}[]
\NormalTok{ma[}\DecValTok{1}\OperatorTok{:}\DecValTok{5}\NormalTok{, }\DecValTok{1}\OperatorTok{:}\DecValTok{3}\NormalTok{]}
\end{Highlighting}
\end{Shaded}

\begin{verbatim}
## An object of class MArray
##  5 features by 3 samples.
\end{verbatim}

\hypertarget{implement-the-summary-of-a-sample}{%
\subsection{Implement the summary of a
sample}\label{implement-the-summary-of-a-sample}}

\begin{Shaded}
\begin{Highlighting}[]
\KeywordTok{isGeneric}\NormalTok{(}\StringTok{"summary"}\NormalTok{)}
\end{Highlighting}
\end{Shaded}

\begin{verbatim}
## [1] FALSE
\end{verbatim}

\begin{Shaded}
\begin{Highlighting}[]
\KeywordTok{setMethod}\NormalTok{(}\StringTok{"summary"}\NormalTok{, }\StringTok{"MArray"}\NormalTok{, }
          \ControlFlowTok{function}\NormalTok{(object, sample) \{}
              \KeywordTok{summary}\NormalTok{(object}\OperatorTok{@}\NormalTok{marray[, sample])}
\NormalTok{          \}          )}
\end{Highlighting}
\end{Shaded}

\begin{Shaded}
\begin{Highlighting}[]
\CommentTok{# get the summary}
\KeywordTok{summary}\NormalTok{(ma, }\StringTok{"D"}\NormalTok{)}
\end{Highlighting}
\end{Shaded}

\begin{verbatim}
##    Min. 1st Qu.  Median    Mean 3rd Qu.    Max. 
##   6.478  10.416  11.999  11.974  13.241  17.641
\end{verbatim}

\hypertarget{for-a-condition-in-probe-get-the-value-for-any-probe}{%
\subsection{\texorpdfstring{For a condition in \texttt{probe} get the
value for any
probe}{For a condition in probe get the value for any probe}}\label{for-a-condition-in-probe-get-the-value-for-any-probe}}

\begin{Shaded}
\begin{Highlighting}[]
\CommentTok{# for the condition "WT" in the sample annotations, }
\CommentTok{# get the array values for probe8}
\NormalTok{wt <-}\StringTok{ }\NormalTok{maexp}\OperatorTok{$}\NormalTok{pmeta[, }\StringTok{"condition"}\NormalTok{] }\OperatorTok{==}\StringTok{ "WT"}
\NormalTok{maexp}\OperatorTok{$}\NormalTok{marray[}\StringTok{"probe8"}\NormalTok{, wt]}
\end{Highlighting}
\end{Shaded}

\begin{verbatim}
##         A         B         C 
## 11.383533  9.226863  9.309726
\end{verbatim}

\begin{Shaded}
\begin{Highlighting}[]
\KeywordTok{setGeneric}\NormalTok{(}\StringTok{"getValuesForProbeCondition"}\NormalTok{, }
           \ControlFlowTok{function}\NormalTok{(object, ...) }\KeywordTok{standardGeneric}\NormalTok{(}\StringTok{"getValuesForProbeCondition"}\NormalTok{))}
\end{Highlighting}
\end{Shaded}

\begin{verbatim}
## [1] "getValuesForProbeCondition"
\end{verbatim}

\begin{Shaded}
\begin{Highlighting}[]
\KeywordTok{setMethod}\NormalTok{(}\StringTok{"getValuesForProbeCondition"}\NormalTok{, }\StringTok{"MArray"}\NormalTok{,}
          \ControlFlowTok{function}\NormalTok{(object, probe, condition) \{}
\NormalTok{              cond =}\StringTok{ }\NormalTok{object}\OperatorTok{@}\NormalTok{pmeta[, }\StringTok{"condition"}\NormalTok{] }\OperatorTok{==}\StringTok{ }\NormalTok{condition}
              \KeywordTok{return}\NormalTok{(object}\OperatorTok{@}\NormalTok{marray[probe, cond])}
\NormalTok{          \})}
\end{Highlighting}
\end{Shaded}

\begin{Shaded}
\begin{Highlighting}[]
\KeywordTok{getValuesForProbeCondition}\NormalTok{(ma, }\StringTok{"probe3"}\NormalTok{, }\StringTok{"WT"}\NormalTok{)}
\end{Highlighting}
\end{Shaded}

\begin{verbatim}
##         A         B         C 
## 12.058866  6.547206 11.572435
\end{verbatim}

\hypertarget{get-type-of-conditions-in-the-samples}{%
\subsection{Get type of conditions in the
samples}\label{get-type-of-conditions-in-the-samples}}

\begin{Shaded}
\begin{Highlighting}[]
\KeywordTok{setGeneric}\NormalTok{(}\StringTok{"getSampleConditions"}\NormalTok{, }\ControlFlowTok{function}\NormalTok{(object, ...) }
    \KeywordTok{standardGeneric}\NormalTok{(}\StringTok{"getSampleConditions"}\NormalTok{))}
\end{Highlighting}
\end{Shaded}

\begin{verbatim}
## [1] "getSampleConditions"
\end{verbatim}

\begin{Shaded}
\begin{Highlighting}[]
\KeywordTok{setMethod}\NormalTok{(}\StringTok{"getSampleConditions"}\NormalTok{, }\StringTok{"MArray"}\NormalTok{, }
          \ControlFlowTok{function}\NormalTok{(object) \{}
              \KeywordTok{unique}\NormalTok{(object}\OperatorTok{@}\NormalTok{pmeta}\OperatorTok{$}\NormalTok{condition)}
\NormalTok{          \})}
\end{Highlighting}
\end{Shaded}

\begin{Shaded}
\begin{Highlighting}[]
\KeywordTok{getSampleConditions}\NormalTok{(ma)}
\end{Highlighting}
\end{Shaded}

\begin{verbatim}
## [1] WT  MUT
## Levels: MUT WT
\end{verbatim}

\hypertarget{get-type-of-pathways-in-the-probes}{%
\subsection{Get type of pathways in the
probes}\label{get-type-of-pathways-in-the-probes}}

\begin{Shaded}
\begin{Highlighting}[]
\KeywordTok{setGeneric}\NormalTok{(}\StringTok{"getProbePathways"}\NormalTok{, }\ControlFlowTok{function}\NormalTok{(object, ...)}
           \KeywordTok{standardGeneric}\NormalTok{(}\StringTok{"getProbePathways"}\NormalTok{))}
\end{Highlighting}
\end{Shaded}

\begin{verbatim}
## [1] "getProbePathways"
\end{verbatim}

\begin{Shaded}
\begin{Highlighting}[]
\KeywordTok{setMethod}\NormalTok{(}\StringTok{"getProbePathways"}\NormalTok{, }\StringTok{"MArray"}\NormalTok{, }
          \ControlFlowTok{function}\NormalTok{(object) \{}
              \KeywordTok{unique}\NormalTok{(object}\OperatorTok{@}\NormalTok{fmeta}\OperatorTok{$}\NormalTok{pathway)}
\NormalTok{          \})}
\end{Highlighting}
\end{Shaded}

\begin{Shaded}
\begin{Highlighting}[]
\KeywordTok{getProbePathways}\NormalTok{(ma)}
\end{Highlighting}
\end{Shaded}

\begin{verbatim}
## [1] L Z A D M O P
## Levels: A D L M O P Z
\end{verbatim}

\hypertarget{introspection}{%
\subsection{Introspection}\label{introspection}}

\begin{Shaded}
\begin{Highlighting}[]
\KeywordTok{slotNames}\NormalTok{(ma)}
\end{Highlighting}
\end{Shaded}

\begin{verbatim}
## [1] "marray" "fmeta"  "pmeta"
\end{verbatim}

\begin{Shaded}
\begin{Highlighting}[]
\KeywordTok{getClass}\NormalTok{(}\StringTok{"MArray"}\NormalTok{)}
\end{Highlighting}
\end{Shaded}

\begin{verbatim}
## Class "MArray" [in ".GlobalEnv"]
## 
## Slots:
##                                        
## Name:      marray      fmeta      pmeta
## Class:     matrix data.frame data.frame
\end{verbatim}

\begin{Shaded}
\begin{Highlighting}[]
\KeywordTok{showMethods}\NormalTok{(}\DataTypeTok{classes =} \StringTok{"MArray"}\NormalTok{)}
\end{Highlighting}
\end{Shaded}

\begin{verbatim}
## 
## Function ".DollarNames":
##  <not an S4 generic function>
## Function: [ (package base)
## x="MArray"
## 
## 
## Function "complete":
##  <not an S4 generic function>
## Function: fmeta (package .GlobalEnv)
## object="MArray"
## 
## 
## Function "formals<-":
##  <not an S4 generic function>
## 
## Function "functions":
##  <not an S4 generic function>
## Function: getProbePathways (package .GlobalEnv)
## object="MArray"
## 
## Function: getSampleConditions (package .GlobalEnv)
## object="MArray"
## 
## Function: getValuesForProbeCondition (package .GlobalEnv)
## object="MArray"
## 
## Function: initialize (package methods)
## .Object="MArray"
##     (inherited from: .Object="ANY")
## 
## Function: marray (package .GlobalEnv)
## object="MArray"
## 
## Function: pmeta (package .GlobalEnv)
## object="MArray"
## 
## 
## Function "prompt":
##  <not an S4 generic function>
## Function: show (package methods)
## object="MArray"
## 
## Function: summary (package base)
## object="MArray"
\end{verbatim}


\end{document}
